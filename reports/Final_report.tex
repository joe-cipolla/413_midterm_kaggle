\documentclass[floatsintext,man]{apa6}

\usepackage{amssymb,amsmath}
\usepackage{ifxetex,ifluatex}
\usepackage{fixltx2e} % provides \textsubscript
\ifnum 0\ifxetex 1\fi\ifluatex 1\fi=0 % if pdftex
  \usepackage[T1]{fontenc}
  \usepackage[utf8]{inputenc}
\else % if luatex or xelatex
  \ifxetex
    \usepackage{mathspec}
    \usepackage{xltxtra,xunicode}
  \else
    \usepackage{fontspec}
  \fi
  \defaultfontfeatures{Mapping=tex-text,Scale=MatchLowercase}
  \newcommand{\euro}{€}
\fi
% use upquote if available, for straight quotes in verbatim environments
\IfFileExists{upquote.sty}{\usepackage{upquote}}{}
% use microtype if available
\IfFileExists{microtype.sty}{\usepackage{microtype}}{}

% Table formatting
\usepackage{longtable, booktabs}
\usepackage{lscape}
% \usepackage[counterclockwise]{rotating}   % Landscape page setup for large tables
\usepackage{multirow}		% Table styling
\usepackage{tabularx}		% Control Column width
\usepackage[flushleft]{threeparttable}	% Allows for three part tables with a specified notes section
\usepackage{threeparttablex}            % Lets threeparttable work with longtable

% Create new environments so endfloat can handle them
% \newenvironment{ltable}
%   {\begin{landscape}\begin{center}\begin{threeparttable}}
%   {\end{threeparttable}\end{center}\end{landscape}}

\newenvironment{lltable}
  {\begin{landscape}\begin{center}\begin{ThreePartTable}}
  {\end{ThreePartTable}\end{center}\end{landscape}}




% The following enables adjusting longtable caption width to table width
% Solution found at http://golatex.de/longtable-mit-caption-so-breit-wie-die-tabelle-t15767.html
\makeatletter
\newcommand\LastLTentrywidth{1em}
\newlength\longtablewidth
\setlength{\longtablewidth}{1in}
\newcommand\getlongtablewidth{%
 \begingroup
  \ifcsname LT@\roman{LT@tables}\endcsname
  \global\longtablewidth=0pt
  \renewcommand\LT@entry[2]{\global\advance\longtablewidth by ##2\relax\gdef\LastLTentrywidth{##2}}%
  \@nameuse{LT@\roman{LT@tables}}%
  \fi
\endgroup}


\ifxetex
  \usepackage[setpagesize=false, % page size defined by xetex
              unicode=false, % unicode breaks when used with xetex
              xetex]{hyperref}
\else
  \usepackage[unicode=true]{hyperref}
\fi
\hypersetup{breaklinks=true,
            pdfauthor={},
            pdftitle={Mid Term Report},
            colorlinks=true,
            citecolor=blue,
            urlcolor=blue,
            linkcolor=black,
            pdfborder={0 0 0}}
\urlstyle{same}  % don't use monospace font for urls

\setlength{\parindent}{0pt}
%\setlength{\parskip}{0pt plus 0pt minus 0pt}

\setlength{\emergencystretch}{3em}  % prevent overfull lines


% Manuscript styling
\captionsetup{font=singlespacing,justification=justified}
\usepackage{csquotes}
\usepackage{upgreek}



\usepackage{tikz} % Variable definition to generate author note

% fix for \tightlist problem in pandoc 1.14
\providecommand{\tightlist}{%
  \setlength{\itemsep}{0pt}\setlength{\parskip}{0pt}}

% Essential manuscript parts
  \title{Mid Term Report}

  \shorttitle{Predict 413}


  \author{Rahul Sangole\textsuperscript{1}}

  % \def\affdep{{""}}%
  % \def\affcity{{""}}%

  \affiliation{
    \vspace{0.5cm}
          \textsuperscript{1} Northwestern University  }



  \abstract{This reports the work done for the Predict 413 Section 55 Midterm
assignment, Summer 2018.}
  
\usepackage[titles]{tocloft}
\cftpagenumbersoff{figure}
\renewcommand{\cftfigpresnum}{\itshape\figurename\enspace}
\renewcommand{\cftfigaftersnum}{.\space}
\setlength{\cftfigindent}{0pt}
\setlength{\cftafterloftitleskip}{0pt}
\settowidth{\cftfignumwidth}{Figure 10.\qquad}



  \raggedbottom

\usepackage{amsthm}
\newtheorem{theorem}{Theorem}[section]
\newtheorem{lemma}{Lemma}[section]
\theoremstyle{definition}
\newtheorem{definition}{Definition}[section]
\newtheorem{corollary}{Corollary}[section]
\newtheorem{proposition}{Proposition}[section]
\theoremstyle{definition}
\newtheorem{example}{Example}[section]
\theoremstyle{definition}
\newtheorem{exercise}{Exercise}[section]
\theoremstyle{remark}
\newtheorem*{remark}{Remark}
\newtheorem*{solution}{Solution}
\begin{document}

\maketitle

\setcounter{secnumdepth}{0}



\section{Overview of Methodology
Used}\label{overview-of-methodology-used}

The input data consists of four files, detailing information about the
shops, items, item categories and the daily sales information over Jan
2013 to Oct 2015. The textual data is in Russian, which is converted to
English first. After data preparation activities, extensive Exploratory
Data Analysis (EDA) is performed on the data. This involved univariate
numerical and graphical summaries, multivariate graphical and numerical
summaries, and unsupervised time series clustering. The EDA results in
insights which feed in to the data preparation step, feature engineering
step as well as modeling and post processing steps. A total of 8 models
are built including one ensemble model. All the models follow a top-down
approach as described later. Some standard model evaluation metrics are
used during the model building process, though the final model selected
is dependent on the Kaggle score.

\section{Data Preparation \& Exploratory Data
Analysis}\label{data-preparation-exploratory-data-analysis}

\subsection{Translation}\label{translation}

The text fields of the input dataset are in Russian. The first step is
to convert these fields into English. This is performed passing the
Russian text to a Google Translate API {[}**{]} via an R script running
on an Amazon Web Services RStudio server. Since a free version of the
Translate API is used, the translation activity runs at a speed of
\textasciitilde{}1 translation per second resulting in an end-to-end
runtime of \textasciitilde{}3 hours. Post translation, the shop meta
data, item category metadata and item level metadata becomes readable
and allows for feature engineering. Table 1 shows a sample of
\texttt{items} translated into English.

\begin{table}[H]

\caption{\label{tab:unnamed-chunk-1}Sample Items}
\centering
\resizebox{\linewidth}{!}{\begin{tabular}[t]{lll}
\toprule
item\_name & item\_id & item\_category\_id\\
\midrule
Risen [PC, Digital Version] & 6146 & 21\\
* LINE OF DEATH D & 11 & 40\\
OTHER WORLD (region) & 11319 & 62\\
1C: Audiobooks. Mandelstam Osip. Egyptian Brand (Jewel) & 311 & 45\\
ARMSTRONG LOUIS Ambassador Of Jazz Box 10CD + Book (box) & 1433 & 31\\
\bottomrule
\end{tabular}}
\end{table}

\subsection{Feature Engineering}\label{feature-engineering}

\subsubsection{Categorical Predictors}\label{categorical-predictors}

The \texttt{item\ category} can be split into two levels of information,
as shown in table 2. \texttt{itemcat\_lvl1} is a higher level
categorization consisting of 21 different levels \emph{(Cinema, Games,
PC Games, Music, Gifts, Movies, Accessories, Books, Programs, Payment
Cards, Game Consoles, Office, Elements of a food, Clean media (piece),
Delivery of goods, Tickets (figure), Official, Clean carriers (spire),
Android games, MAC Games, PC)}, while \texttt{itemcat\_lvl2} is a lower
level categorization consisting of 62 different levels. A sample is
shown in table 2.

\begin{table}[H]

\caption{\label{tab:unnamed-chunk-2}Sample Item Categories}
\centering
\resizebox{\linewidth}{!}{\begin{tabular}[t]{llll}
\toprule
item\_category\_name & itemcat\_lvl1 & itemcat\_lvl2 & item\_category\_id\\
\midrule
Books - Artbook, encyclopedia & Books & Artbook, encyclopedia & 42\\
Games - Accessories for games & Games & Accessories for games & 25\\
Payment Cards - Windows (Digital) & Payment Cards & Windows (Digital) & 35\\
Music - CD of branded production & Music & CD of branded production & 56\\
Accessories - XBOX 360 & Accessories & XBOX 360 & 6\\
\bottomrule
\end{tabular}}
\end{table}

The \texttt{shops} table consists of some location information about the
shops. This is split into two categorical predictors as well.
\texttt{loc\_lvl} is a higher level categorization consisting of 32
different levels, while \texttt{loc\_lvl1} is a deeper categorization
consisting of 56 levels. A sample is shown in table 3.

\begin{table}[H]

\caption{\label{tab:unnamed-chunk-3}Sample Shops}
\centering
\fontsize{11}{13}\selectfont
\begin{tabular}[t]{lll}
\toprule
loc\_lvl1 & loc\_lvl2 & shop\_id\\
\midrule
Moscow & TC Perlovsky & 30\\
Krasnoyarsk & Shopping center June & 18\\
Moscow & TC Budenovskiy (pav.K7) & 24\\
RostovNaDonu & TRC Megacenter Horizon & 39\\
Volzhsky & shopping center Volga Mall & 4\\
\bottomrule
\end{tabular}
\end{table}

\subsubsection{Calendar Related
Predictors}\label{calendar-related-predictors}

Temporal predictors are appended to the dataset, viz.,

\begin{enumerate}
\def\labelenumi{\arabic{enumi}.}
\tightlist
\item
  \texttt{year}, \texttt{month}, \texttt{week} describing the year,
  month and week of the observation
\item
  \texttt{weekend} is a binary 0/1 variable to account for increased
  sales over a weekend (if any)
\item
  \texttt{ym}, year-month combination
\item
  \texttt{yw}, year-week combination
\item
  \texttt{is\_december}, is a binary 0/1 variable to account for
  increased Christmas / New Year sales (if any)
\end{enumerate}

Russian holiday schedules are downloaded for the years 2013, 2014 and
2015 from \enquote{Holidays in Russia,
'\url{https://www.timeanddate.com/holidays/russia/2013\#!hol=9}}. These
are filtered to \emph{Official and Non-Working Days} are joined to the
original data.

\begin{verbatim}
## Warning: package 'bindrcpp' was built under R version 3.4.4
\end{verbatim}

\begin{table}[H]

\caption{\label{tab:unnamed-chunk-4}2013 Russian Holidays}
\centering
\begin{tabular}[t]{lll}
\toprule
date & holiday\_name & holiday\_type\\
\midrule
2013-01-01 & New Year's Day & National holiday\\
2013-01-02 & New Year Holiday Week & National holiday\\
2013-01-03 & New Year Holiday Week & National holiday\\
2013-01-04 & New Year Holiday Week & National holiday\\
2013-01-07 & Orthodox Christmas Day & National holiday, Orthodox\\
\addlinespace
2013-01-08 & New Year Holiday Week & National holiday\\
2013-02-23 & Defender of the Fatherland Day & National holiday\\
2013-03-08 & International Women's Day & National holiday\\
2013-05-01 & Spring and Labor Day & National holiday\\
2013-05-02 & Spring and Labor Day Holiday & National holiday\\
\addlinespace
2013-05-03 & Spring and Labor Day Holiday & National holiday\\
2013-05-09 & Victory Day & National holiday\\
2013-05-10 & Defender of the Fatherland Day holiday & National holiday\\
2013-05-10 & Victory Day Holiday & National holiday\\
2013-06-12 & Russia Day & National holiday\\
\addlinespace
2013-09-01 & Day of Knowledge & De facto holiday\\
2013-11-04 & Unity Day & National holiday\\
2013-12-30 & New Year Holiday Week & De facto holiday\\
\bottomrule
\end{tabular}
\end{table}

\subsection{Data Exploration}\label{data-exploration}

\subsubsection{Time Series Exploration}\label{time-series-exploration}

\subsubsection{t-SNE}\label{t-sne}

\subsubsection{Time Series Clustering}\label{time-series-clustering}

\subsubsection{Oddities}\label{oddities}

\subsection{Data Cleansing}\label{data-cleansing}

removing -ve counts closed shops spikey stuff

\section{Modeling Details}\label{modeling-details}

\subsection{High level view of
approaches}\label{high-level-view-of-approaches}

\subsection{Detailed Model List}\label{detailed-model-list}

\subsection{Code Details}\label{code-details}

\section{Results Summary}\label{results-summary}

\subsection{Performance Evaluation}\label{performance-evaluation}

\section{Challenges}\label{challenges}

\newpage

\section{R Packages Used}\label{r-packages-used}

\newpage

\section{References}\label{references}

\begin{verbatim}
## Warning in readLines(file): incomplete final line found on 'r-
## references.bib'
\end{verbatim}

\begingroup
\setlength{\parindent}{-0.5in} \setlength{\leftskip}{0.5in}

\hypertarget{refs}{}

\endgroup



\clearpage
\renewcommand{\listfigurename}{Figure captions}
\listoffigures



\end{document}
